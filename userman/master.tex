\documentclass[11pt]{article}
\usepackage{z-eves}
\usepackage{alltt}
\newcommand{\Jaza}{Jaza}
\newcommand{\url}[1]{\texttt{#1}}
\newcommand{\jazainput}[1]{{\ttfamily\itshape #1}}
\newenvironment{Advanced}{\textbf{Advanced Topic: }}%
                         {\par\vskip-\parskip\textbf{End of Advanced Topic.}}
\parindent 0pt
\parskip 1.5ex plus 3pt minus 2pt

\title{Jaza User Manual and Tutorial}
\author{Mark Utting \\ 
  The University of Waikato \\
  Hamilton, New Zealand \\
  Email: \texttt{marku@cs.waikato.ac.nz}}

\begin{document}
\maketitle

% Rationale and Overview
\emph{Jaza} stands for \emph{`Just Another Z Animator'}.
It is an open-source animator for the Z specification language.
It is written in Haskell, which is a modern strongly-typed
functional programming language.\footnote{See
  \url{http://www.haskell.org}.}
The current version of {\Jaza} has several major
limitations:
\begin{itemize}
\item it supports only {\LaTeX} Z notation for input and output.
  This is not very readable for Z novices, but good for Z experts 
  who want to cut and paste between {\Jaza} and the {\LaTeX}
  source of their Z specifications.
  However, it might be nice to support the Z standard email syntax
  or provide a formatted graphical view of the output, as well
  as supporting {\LaTeX} notation.
\item it does not support all of Z (generics and axiomatic definitions
  are the main omissions).  See Section~\ref{sec:limitations} for more details.
\item it does not have a graphical user interface!
\end{itemize}

The Community Z Tools (CZT) project is developing a new suite
of Z tools, including an animator, that will fix these restrictions.
See \url{http://czt.sourceforge.net} for details.


\section{Introduction and Motivation} \label{sec:intro}

The goal of writing a Z specification for a system is to
give a precise description of some aspect of the system's behaviour.
The specification can be checked to ensure that it contains no
obvious errors (syntax errors, type errors, etc.).
However, this does not guarantee that the specification describes
the system that we really want.  It is almost as easy to make
mistakes or misunderstanding in specifications as it is in programs.
There are four main ways of validating a specification to check
that it correctly captures our requirements: 

\begin{description}
\item[Inspection:] Z experts, preferably together with domain experts, 
  read the specification and manually cross-check it against the
  requirements.
\item[Proof:] Important requirements are formalized and then we prove that
  they follow from the specification.  There are several good proof tools
  available to help with such proofs.\footnote{See
    \url{http://archive.comlab.ox.ac.uk/z.html} for a list of Z tools.}
\item[Execution:] Some Z specifications may happen to be written in
  a form that is executable.  That is, given some input values for
  an operation, a tool such as {\Jaza} can calculate the output values
  automatically.  In this case, it is possible for users to interact
  with the specified system and try various sequences of operations,
  checking to see if they have the desired behaviour.
  However, not all Z specifications are executable in this fashion,
  and it is often better style to write specifications in a
  non-executable fashion~\cite{hayes:specs-not-exec}.
\item[Testing:] If values are given for all the inputs and outputs
  of a specified operation, then {\Jaza} can usually evaluate the
  specification of that operation to either true or false.
  So a useful validation technique is to provide for each operation
  several positive test sets (which should cause the specification to
  evaluate to true)
  as well as several negative test sets (which should evaluate to false).
  This is a powerful validation technique, since it allows the behaviour of
  the operation to be bounded on both sides.
\end{description}

Ideally, all these techniques should be used together to validate a
specification.  Inspection is a cost-effective way of uncovering errors and
is usually cheaper to apply than proof.  Proof is the only technique that
can give guarantees about all possible behaviours of the system.  The usual
caveats about testing apply to the last two techniques: they can show the
presence of errors in the specification, but cannot guarantee the absence
of all errors.

{\Jaza} is aimed at supporting the last two validation techniques:
execution and testing.  It is designed to complement a type-checker
and prover, such as Z/EVES~\cite{saaltink:zeves-system}, by providing
more efficient and convenient evaluation of schemas on ground data values.

{\Jaza} can also be used as a \emph{desk-calculator} for Z expressions
and predicates.  This is useful for students who are learning
the Z toolkit and wish to experiment with its operators and
data types (sets, relations, functions, sequences etc.).
The following sections describe how {\Jaza} supports each of
these techniques.


\section{Starting {\Jaza}}
\label{sec:start}

{\Jaza} can be used with either the Hugs interpreter, or the GHC
compiler (see \url{http://www.haskell.org} to obtain these systems).
{\Jaza} will be much faster if you use GHC, but the Hugs version is
quite usable for most Z specifications, though you will need to
increase the heap size for most specifications.  

NOTE: To set the heap size with Hugs, you can start Hugs with a
\texttt{-h20m} flag, or set the HUGSFLAGS environment variable to
\texttt{"-h20m"}.  The 20m means 20 million \emph{cells}---since cells
are usually two integers each, this is about 160 megabytes on a 32 bit
machine.  On Windows systems, you can type \texttt{:set -h20m} within
hugs (which saves that setting in the Windows registry), and then
restart hugs for the new setting to take effect.  With the binary
version of Jaza, which is compiled with GHC, the heap will start small
(which means lots of garbage collections) then grow as big as
necessary.  If you want to experiment with a large fixed-size heap (to
reduce the initial garbage collections), then you can either set the
environment variable \verb!GHCRTS='-M500m'! (for 500 megabytes) or
adding these flags to the command line \verb!+RTS -M500m -RTS!.

If you have a version of {\Jaza} that has been compiled with GHC,
you can start Jaza by clicking on the \texttt{jaza.exe} program under
Windows 95/NT, or just typing \texttt{jaza} on Linux or UNIX systems
(you may need to set your PATH environment variable to include
the {\Jaza} directory).  For example, under Linux:
\begin{verbatim}
    export PATH=$PATH:/Directory-Where-Jaza-Is-Installed
    jaza
\end{verbatim}
%  $ match above dollar!

If you are using the Hugs interpreter under Windows 95/NT, you should be
able to just double click on the \texttt{TextUI.hs} file in the {\Jaza}
directory.  Under Linux/UNIX, the easiest way of starting
Jaza is to use the command line version of Hugs:
\begin{verbatim}
    export HUGSFLAGS="-h20m"
    runhugs /Directory-Where-Jaza-Is-Installed/TextUI.hs
\end{verbatim}
An alternative way is to start up the interactive version of Hugs, 
then load the \texttt{TextUI.hs} module, then type \texttt{'main'}.
\begin{verbatim}
    export HUGSFLAGS="-h20m"
    hugs
    Prelude> :load TextUI
    Main> main
    Welcome to Jaza, version ...
    JAZA> 
\end{verbatim}

Whichever way you choose, you should see the \verb!JAZA>! prompt.
Typing \verb!help! at this prompt will show a summary of the
{\Jaza} commands.  To exit {\Jaza}, just type \verb!quit!.
In this manual, all Jaza input and output will
be shown in typewriter font.
%\jazainput{italic typewriter} font, and output will be shown 
%in \texttt{ordinary typewriter} font.

\section{Evaluating Expressions}
\label{sec:expr}

To evaluate a Z expression, you use the \verb!eval E! command.
The expression \verb!E! must be written in LaTeX notation.
It can spread across multiple lines if the characters after the \verb!eval! 
command contain more opening brackets (\verb!(!, \verb![!, or \verb!{!)
than closing brackets.  In this case, {\Jaza} will prompt for additional
lines of input, until the right number of closing brackets are found.
(During the entry of multi-line input, the {\Jaza} prompt changes to
six spaces, so that the input lines look properly aligned,
but can still be selected by dragging a mouse over the whole
paragraph.  This makes it easy to cut and paste large expressions
between {\Jaza} and other documents.)

For example:
\begin{verbatim}
    JAZA> eval 123456 * 1000000
    123456000000
    JAZA> eval \{ x : 0 \upto 10 
                  @ x*x \}
    \{0, 1, 4, 9, 16, 25, 36, 49, 64, 81, 100\}
\end{verbatim}

Similarly, to evaluate a Z predicate, you use the 
\verb!evalp P! command:
\begin{verbatim}
    JAZA> evalp 12 < 12
    False
    JAZA> evalp 2 \in \dom \langle 3, 5, 7 \rangle
    True
\end{verbatim}


\emph{Warning:} {\Jaza} currently does not perform Z 
typechecking on the Z constructs that you enter.  
If you enter incorrectly typed terms, it will either return 
with an error (not always a type error), or it will give a 
sensible result according to untyped set theory,
even though that result may not have a legal Z type.
For example:
\begin{verbatim}
    JAZA> eval \{3,3,2\} \cup \{ (1,4) \}
    \{2, 3, (1, 4)\}
    JAZA> evalp 3 \in 5
    Coercion Problem: cannot convert: ZInt 5, into finite set
\end{verbatim}


\section{Entering Specifications}
\label{sec:enter}

You can enter Z specifications into {\Jaza} in two ways:
\begin{itemize}
\item Type each Z paragraph directly into the {\Jaza} prompt.
  Each paragraph can span several lines, but must start 
  with \verb!\begin{...}! and end with a matching \verb!\end{...}!,
  where the \verb!...! can be any Z LaTeX environment, such as 
  \verb!zed!, \verb!schema!, \verb!axdef! or \verb!gendef!. 
  Syntax errors will not be detected until you have correctly
  entered the \verb!\end{...}! command.
  
\item Load a specification from a file.  In this case, {\Jaza}
  ignores all text in the file, except for the Z paragraphs
  such as \verb!\begin{zed}! \ldots \verb!\end{zed}!.
  Each load command simply adds more definitions onto the stack
  of definitions within {\Jaza}.
\end{itemize}

The recommended style is to write your Z specification in one or
more separate files, use a typechecker (such as Z/EVES) to check
the specification, then the \verb!load! command of {\Jaza} to load
the files in the right order (so that names are defined before they
are used).

{\Jaza} provides several commands for inspecting and undoing a
specification that you have entered:
\begin{description}
\item['\texttt{show}':] This displays a summary of the whole
  specification, with one definition per line.
\item['\texttt{show} N':] This displays the internal 'unfolded'
  form of the schema or defined name 'N'.
\item['\texttt{pop}':] This removes the most recently entered 
  Z definition.
\item['\texttt{reset}':] This removes the entire specification
  and resets {\Jaza} to its starting state.
\end{description}

For a running example in this manual, I use the standard
BirthdayBook system from Spivey~\cite{spivey:z-notation2}.

Here is an example of entering the first two paragraphs
directly into {\Jaza}.
\begin{verbatim}
    JAZA> \begin{zed}
            [NAME, DATE] 
          \end{zed}
    Added 2 definitions.
    JAZA> \begin{schema}{BirthdayBook}
              known: \power NAME \\
              birthday: NAME \pfun DATE
          \where
              known=\dom birthday
          \end{schema}
    Added 1 definition.
    JAZA> 
\end{verbatim}

Tiring of this, let's load the whole specification
from a file called \verb!bbook.zed! (this is included in the
Jaza distribution, either at the top level or in the userman directory).
Since this file contains the above definitions (of $NAME$, $DATE$ and
$BirthdayBook$), we must do a \verb!reset! command first -- otherwise
Jaza would complain that those names are being redefined.
\begin{verbatim}
    JAZA> reset
    Specification is now empty.
    JAZA> load userman/bbook.zed
    load userman/bbook.zed
    Loading userman/bbook.zed...
    Added 13 definitions.
    JAZA> 
\end{verbatim}

We can now see the expanded forms of the schemas, like this:
\begin{verbatim}
    JAZA> show InitBirthdayBook
    [known:\power NAME; birthday:NAME \pfun DATE; 
     known = \dom birthday; known = \{\}]
\end{verbatim}

\begin{Advanced}
Note that schemas are displayed in horizontal form (\verb![...]!)
and the contents are simply a list of declarations and predicate tests,
with no clear separation between the declaration and predicate parts
of the schema.  Internally, {\Jaza} uses a generalized schema notation
which allows predicates and declarations to be intermixed, so that
some predicate tests can be moved in amongst the declarations.
Since schemas are evaluated from left to right, this is one
technique that is used to improve evaluation efficiency.
Schemas are 'optimized' just before being executed, so this
output from the \verb!show! command shows the schema in
unoptimized form.  
It is sometimes interesting to see how a schema has been
optimized -- you can do this with the 
\verb!why! command after the execution of the schema.
\end{Advanced}


\section{Testing Schemas}
\label{sec:test}

After Spivey defines the $BirthdayBook$ state schema,
he gives some example values that satisfy the schema:
\[
        known = \{\,{\rm John, Mike, Susan}\,\} \\
\also
        birthday = \{\,\vtop{\halign{\strut#\hfil&${}\mapsto{}$#\hfil\cr
                        John&  25--Mar,\cr
                        Mike&  20--Dec,\cr
                        Susan& 20--Dec\,\}.\cr}}
\]

This is a useful validation check, because it ensures that
the sample data satisfies the state invariant.  However,
the syntax shown here is somewhat informal, since $John$
and \emph{25--Mar} \emph{etc.} are not legal Z values (no members
of the $NAME$ and $DATE$ given sets have been declared).

{\Jaza} provides a convenient way of creating members of
any given set: simply use any string surrounded by double quotes!

To check the above example values, we could use the
\verb!do! command to evaluate a schema expression, like this:
\begin{verbatim}
    JAZA> do [BirthdayBook |
          \t1     known = \{\,"John", "Mike", "Susan"\,\} \land \\
          \t1     birthday = \{\,"John"  \mapsto "25-Mar", \\
          \t4                    "Mike"  \mapsto "20-Dec", \\
          \t4                    "Susan" \mapsto "20-Dec"\,\} ]
    \lblot birthday==\{("John", "25-Mar"), ("Mike", "20-Dec"),
                    ("Susan", "20-Dec")\},
            known==\{"John", "Mike", "Susan"\} \rblot
\end{verbatim}

This displays one possible binding that satisfies the schema
(in this case, it is the only such binding).  
The command \verb!do S! essentially returns the set of all bindings
that satisfy the schema expression \verb!S!.  However, since that
set of bindings is typically infinite, it displays only the first
binding and leaves any remaining bindings unevaluated until you
ask to see them.  If a schema has several solutions, we can step 
through them using the \verb!next! and \verb!prev! commands.  

To illustrate this, let us use another schema.  
It finds factors of $x$.
\begin{schema}{FactX}
  x, top : \nat \\
  factors : \power \nat
\where
  factors = \{f:0 \upto x | (\exists g:0 \upto x @ f*g=x) \} \\
  x < top
\end{schema}

If we try to find a binding for $FactX$, with $x=12$, using the command
\verb!do [FactX|x=12]!, {\Jaza} gives an error, because $top$
is an arbitrary natural number, so the search space is
too large (infinite, in this case).
\begin{verbatim}
    JAZA> do [FactX | x=12]
    Problem: cannot convert: \{x:\num | x \geq 13\}
                      into finite set
\end{verbatim}

This illustrates how, to limit memory required for each evaluation and
ensure quick response times, {\Jaza} places limits on the maximum size of
any set (\texttt{maxset}) and the maximum search space (\texttt{maxsearch}).  
When any of these limits are exceeded, a failure message is returned
instead of a normal result.  The current values of these limits
can be viewed and changed via the \texttt{set} command.

To see more clearly why this schema was too difficult to
animate, we can use the \verb!why! command to display the
optimised form of the schema which Jaza tried to execute.
\begin{verbatim}
JAZA> why
\begin{schema}{[FactX | x=12]}
1  x==12:\nat
2  factors==\{f:0 \upto x; g:0 \upto x; f * g = x @ f\}:\power \nat
3  top:\bigcap \{\nat, (_ ZGreaterThan x)\}
\end{schema}
The maximum number of true constraints was 2.
\end{verbatim}

Jaza tries to execute this sequence from top to bottom (and left to right
within each line), 
so the initial $x==12:\nat$ and $factors==\ldots:\power \nat$ 
lines mean that it has calculated unique solutions for these two variables
(the $:Type$ part checks that the solution has the correct type).
However, line 3 is trying to choose a value of
$top$ from the set $\nat \cap (12 \upto \inf)$, which is
an infinite set.  This shows that we need to give more
constraints on $top$, or specify its value.  The last line
of output tells us that Jaza managed to find solutions to
the first two constraints, but not the third one (the $top:\ldots$).

For example, if we set top to 1000, the search succeeds.
\begin{verbatim}
    JAZA> do [FactX | x=12 \land top=1000]
    \lblot factors==\{0\}, top==1000, x==12 \rblot
\end{verbatim}

A more interesting experiment is to try searching for
the factors of an arbitrary $x$.
For example, if we set top to 10, Jaza chooses an
arbitrary $x$ within the allowable range ($0 \upto 10$)
and displays its factors:
\begin{verbatim}
JAZA> do [FactX | top=10]
    \lblot factors==\{0\}, top==10, x==0 \rblot
\end{verbatim}

Since this schema has more than one solution, we can step through them
using \verb!next! and \verb!prev!, or use \verb!state N! to jump
to the N'th solution (if there are less than N solutions, it stops
at the last solution).
\begin{verbatim}
    JAZA> next
    \lblot factors==\{1\}, top==10, x==1 \rblot
    JAZA> next
    \lblot factors==\{1, 2\}, top==10, x==2 \rblot
    JAZA> state 1000
    No more solutions
    JAZA> curr
    \lblot factors==\{1, 3, 9\}, top==10, x==9 \rblot
    JAZA> prev
    \lblot factors==\{1, 2, 4, 8\}, top==10, x==8 \rblot
\end{verbatim}

All these solutions have the same $top$ value: 10.
Usually we do not want to see the values of fields that we have
just given exact values for, so we use the Z/EVES \emph{replacement}
operator to set $top$ to $10$.  The term $S[n := E]$ is equivalent
to the schema expression
\[
    [S | n = E] \hide (n).
\]

\begin{verbatim}
    JAZA> do FactX[top := 10]
    \lblot factors==\{0\}, x==0 \rblot
    JAZA> state 8
    \lblot factors==\{1, 2, 4, 8\}, x==8 \rblot
\end{verbatim}

Note that 
the \verb!do! command treats all names equally (it does not depend upon the
$x, x', x?, x!$ naming conventions of Z), so you can use it for a variety
of purposes:
\begin{itemize}
\item to search for all solutions to a schema;
\item to execute a schema `forwards' (specify values for the inputs, then let
  it search for possible output values);
\item to execute a schema `backwards (specify values for the outputs, then
  let it search for possible input values);
\item to test particular input and output values (specify values for 
  every name, and see if those values satisfy the schema or not);
\item to explore special cases of the schema (add predicates that
  restrict the set of solutions).
\end{itemize}

Here is a typical sequence of `tests'
for the $FactX$ schema.  The first five are positive
tests that satisfy the schema predicate, so produce a
single empty binding.  The remainder are negative tests
that contradict the schema predicate, so {\Jaza} correctly
reports that they have no solution.
\begin{verbatim}
    JAZA> % Positive Tests for FactX
    JAZA> do FactX[top := 1, x := 0, factors := \{0\}]
    \lblot \rblot
    JAZA> do FactX[top := 10, x := 0, factors := \{0\}]
    \lblot \rblot
    JAZA> do FactX[top := 10, x := 1, factors := \{1\}]
    \lblot \rblot
    JAZA> do FactX[top := 10, x := 5, factors := \{1,5\}]
    \lblot \rblot
    JAZA> do FactX[top := 10, x := 6, factors := \{1,2,3,6\}]
    \lblot \rblot
    JAZA> % Negative Tests for FactX
    JAZA> do FactX[top := -3]
    No solutions
    JAZA> do FactX[top := 1, x := 1]
    No solutions
    JAZA> do FactX[top := 10, x := 6, factors := \{1,6\}]
    No solutions
    JAZA> do FactX[top := 10, x := 6, factors := \{\}]
    No solutions
    JAZA> do FactX[top := 10, x := 6, factors := \{1,2,3,4,5,6\}]
    No solutions
\end{verbatim}

This style of `testing' schemas is the main purpose of {\Jaza}.
I suggest that every operation schema that you write should have several
positive and negative tests to test its behaviour.  It is
also useful to write some positive and negative tests for your
state schemas, to check that your state invariant is correct.
Of course, your initialisation schema is one such positive test,
but adding some negative tests can be a useful way of checking that the
invariant is strong enough.


\section{Executing Specifications}
\label{sec:execute}

If a specification has an initialization schema (say $Init$) plus
several operation schemas (say $Op1 \ldots OpN$), and
follows the usual Z conventions for sequential systems 
($x/x'$ for initial/final state variables, $x?$ for inputs and $x!$
for outputs), then we can try to `execute' it like this:
\begin{verbatim}
    JAZA> do Init'
    JAZA> ; Op3
    JAZA> ; Op1[in? := 4]
    JAZA> ; Op3
\end{verbatim}

How does this work?  
The essence of `executing' a specification is that we have some
current `system state', and we apply various operation schemas to this
system state to transform it into a new state.  Some of the operations
may take inputs (names ending with $?$) and/or produce outputs (names
ending with $!$), in addition to changing the current state of the system.

In Section~\ref{sec:test} we saw that the \verb!do! command displays
just one binding at a time.  This binding is treated as the 
\emph{current system state} and is used as the input state whenever
we apply an operation schema using the \emph{sequential composition}
command \verb!`;'!.  So the result of the \verb!do Init'! command
is a current system state that contains values for each of the
primed variables in $Init'$.  (If you forget the prime, the state will
contain unprimed variables, which will not be usable as inputs of
the next command.  So when you type \verb!; S! next, you will be 
prompted to enter values for all these state variables!)

The \verb!; S! command sequentially composes the current system state
with S, prompts the user for any remaining inputs (unprimed variables or
variables that end with a \verb!?!), then searches for outputs that
satisfy schema \verb!S!.  The first output binding that is found
becomes the new system state (but alternative output bindings can
be explored using the \verb!next! and \verb!prev! commands as usual).

Let us see how this works using the birthday book specification.
After loading the specification (see Section~\ref{sec:enter}),
we start by finding a solution to the $InitBirthdayBook'$ schema
(the prime ensures that we have an \emph{output} state).
\begin{verbatim}
    JAZA> do InitBirthdayBook'
    \lblot birthday'==\{\}, known'==\{\} \rblot
\end{verbatim}

Now we can use the \verb!`;'! command to sequentially compose
an operation schema with this current system state.  If we
do not specify the input values, the \verb!`;'! command prompts
for them.  Alternatively, we can use \verb!:=! to assign values for some
or all inputs. 
\begin{verbatim}
    JAZA> ; RAddBirthday
        Input name? = "andy"
        Input date? = "1-Jan"
    \lblot birthday'==\{("andy", "1-Jan")\}, known'==\{"andy"\},
           result!==ok \rblot
    JAZA> ; RAddBirthday[name? := "beth"]
        Input date? = "31-Dec"
    \lblot birthday'==\{("andy", "1-Jan"), ("beth", "31-Dec")\},
           known'==\{"andy", "beth"\}, result!==ok \rblot
    JAZA> ; RAddBirthday[name? := "carl", date? := "31-Dec"]
    \lblot birthday'==\{("andy", "1-Jan"), ("beth", "31-Dec"),
                    ("carl", "31-Dec")\},
           known'==\{"andy", "beth", "carl"\}, result!==ok \rblot
\end{verbatim}

After entering this data, we use the $FindBirthday$ operation
to check when various people have birthdays.  As expected,
we find that asking for \verb!"andy"! returns the date \verb!"1-Jan"!,
while asking for \verb!"mark"! causes the $FindBirthday$ operation
to fail because its precondition is false.
\begin{verbatim}
    JAZA> ; FindBirthday
        Input name? = "andy"
    \lblot birthday'==\{("andy", "1-Jan"), ("beth", "31-Dec"),
                    ("carl", "31-Dec")\},
           date!=="1-Jan", known'==\{"andy", "beth", "carl"\} \rblot
    JAZA> ; FindBirthday[name? := "mark"]
    No solutions
\end{verbatim}

When an operation returns no solutions, there is no `current system state'
any more, so we cannot continue composing more operations.  To overcome
this, we could start again (\verb!do InitBirthdayBook'!), or we can
use the \verb!undo! command to step back over the unsuccessful
call to $FindBirthday$ and recover the old system state.

\begin{verbatim}
    JAZA> undo
    undone operation: FindBirthday[name? := "mark"]
    JAZA> curr
    \lblot birthday'==\{("andy", "1-Jan"), ("beth", "31-Dec"),
                    ("carl", "31-Dec")\},
           date!=="1-Jan", known'==\{"andy", "beth", "carl"\} \rblot
\end{verbatim}

If we use the (non-deterministic) $RemindOne$ operation to 
find someone who has a birthday on the 31st of December, we get two
solutions. 

\begin{verbatim}
    JAZA> ; RemindOne[today? := "31-Dec"]
    \lblot birthday'==\{("andy", "1-Jan"), ("beth", "31-Dec"),
                    ("carl", "31-Dec")\},
           card!=="beth", known'==\{"andy", "beth", "carl"\} \rblot
    JAZA> next
    \lblot birthday'==\{("andy", "1-Jan"), ("beth", "31-Dec"),
                    ("carl", "31-Dec")\},
           card!=="carl", known'==\{"andy", "beth", "carl"\} \rblot
    JAZA> next
    No more solutions
\end{verbatim}

Of course, if we use $RRemind$ instead, we see the set of all
people in the first (unique) solution of the operation.
Note that the \verb!next! command leaves the current system state
unchanged when it fails, so the input state to this $RRemind$ call
is the last state shown above.

\begin{verbatim}
    JAZA> ; RRemind[today? := "31-Dec"]
    \lblot birthday'==\{("andy", "1-Jan"), ("beth", "31-Dec"),
                    ("carl", "31-Dec")\},
           cards!==\{"beth", "carl"\}, 
           known'==\{"andy", "beth", "carl"\}, 
           result!==ok \rblot
\end{verbatim}


\section{Limitations of {\Jaza}} \label{sec:limitations}

Generally, {\Jaza} supports most of Spivey Z (second edition),
except for generics, axiomatic definitions and bags.  In addition,
it supports the Standard Z \verb!\lblot...\rblot! notation for
bindings, but does not support all the other features of Standard Z,
such as sections.
Here is a list of other limitations:
\begin{itemize}
\item Input is not type checked.  (You should run specifications
  through an external type checker first).
\item Transitive closure: $R\plus$ is implemented, but $R\star$ is not
  (because it requires type inference to determine the basetype of $R$).
\item Signature compatibility of schemas is more strict in
  {\Jaza} than in standard Z.  The types in the two signatures
  must be written identically for those signatures to be compatible.
\item Theta expressions are not allowed to include schema renaming.
\item There is no way of declaring user-defined infix/prefix/postfix 
  functions or relations.
\item {\Jaza} gives errors for axiomatic declaration.  It is not
  clear what an animator should do with axiomatic declarations in general, 
  because it means that your specification is really a parameterized
  family of specifications, and which member of the family should be animated?
  The usual way to avoid this problem is to replace the axiomatic declaration
  by a definition, while you are animating the specification.
\end{itemize}

\section*{Acknowledgements}

Thanks to Greg Reeve for implementing much of the parser and
pretty-printed.  Thanks to the many students who have used Jaza
and provided valuable feedback and test cases.  Especial thanks
to Yves Ledru for suggestions on how to improve the `why' command
and give more feedback about failed evaluations.

\bibliographystyle{alpha}
\bibliography{spec,thmprove,proceedings}

\appendix

\section*{Appendix B: Summary of {\Jaza} Commands}

Note that all expressions, predicates and schema expressions
must be input in {\LaTeX} format.  The syntax follows Spivey
second edition closely, but adds the Z/EVES replacement
operator: $S[x := E]$.  

The available commands in {\Jaza} include:
\begin{description}
\item[help:] Display a summary of all the {\Jaza} commands.
\item[quit:] Exit the animator.
\item[$\backslash$begin\{\ldots\}\ldots$\backslash$end\{\ldots\}:] 
  Enter a Z paragraph.  The paragraph can spread over several lines,
  because {\Jaza} will continue prompting for input lines until 
  the end command is seen.  
\item[load filename:] Load a Z specification from a file.
\item[show:] Display a summary of all the definitions in the specification.
\item[show N:] Display an unfolded definition of the name N.
\item[pop:] Delete the last Z paragraph entered.
\item[reset:] Reset the whole specification, clearing all previously
  entered paragraphs.
\item[do S:] Execute schema expression S.  This displays the
  a solution to the schema, if one can be found or `No solutions'
  if there are no bindings that satisfy the schema.  It may also
  return an error if S could not be evaluated.  If a solution
  is displayed, it becomes the `current state' of the animator.
\item[next (or 'n'):] Show the next state of current schema.
\item[curr:] Reshow the current state.
\item[prev (or 'p'):] Show the previous state.
\item[state N:] Show the N'th state.
\item[; S:] Compose schema S with the current state.
\item[undo:] Undo the current operation.  This removes the
  `current state', and returns to the previous `current state'
  if there was one.  You can type \texttt{`curr'} to see that state.
  Only one level of undo is supported by default, but you can
  increase this by setting the \texttt{undo} parameter higher
  via the \texttt{set} command.
\item[eval E:] Evaluate expression E.  
\item[evalp P:] Evaluate predicate P.  
\item[why:] Show the optimized form of the current schema operation
  or the preceeding expression or predicate evaluation.
  Given a sequence of declarations, $x_1:T_1;\ldots x_n:T_n$
  (probably interspersed with predicates), evaluation is done
  left-to-right, so the maximum size of the search space is the
  cross-product of the $T_i$ sets.  The optimization attempts to
  reduce the size of each $T_i$ set, and reorder them so that
  smaller sets come first.  This command allows you to see
  how successful the optimization was, and perhaps see why a schema 
  or expression cannot be evaluated.  In this mode, false and true
  predicates usually have explanations attached, to show what predicate
  it was that evaluated to false or true.
\item[checktrue P:] Check that P is true, else give an error.
  This is intended mainly for automated regression testing
  of {\Jaza} itself.
\item[checkundef P:] Check that P is undefined, else give an error.
  This is intended mainly for automated regression testing
  of {\Jaza} itself.
\item[set:] Show the current settings of Jaza, such as the
  maximum size of any finite set and the maximum search space size.
\item[set \emph{Param} \emph{NNN}:] Set the parameter called \emph{Param}
  to the number \emph{NNN}.  Current parameters include:
  \begin{itemize}
  \item \texttt{undo} (the number of undo levels), 
  \item \texttt{maxset} (the maximum size of any finite set), and
  \item \texttt{maxsearch} (the maximum search space).
  \end{itemize}
  For example, $\{x,y,z:1\upto 10 @ x+y+z\}$ has a search space of
  1000, but a set size of less than 30.
  These limits are intended to prevent fruitless infinite searches.
  The default limits are quite small, so you may want to increase them
  if you have a fast machine or lots of memory.  If you increase the
  limits, you may also need to increase the Haskell heap size with the
  \texttt{-hNNN} flag.

  Evaluations that require more resources than these limits allow will return
  an error message instead of a normal result.  The internal engine
  is designed so that these limits can affect completeness (low limits may
  make it impossible to evaluate some expressions), but not correctness
  (for example, reducing the search space will never cause a $\forall$
  or $\exists$ to return $true$ instead of $false$, or vice versa).
\item[echo msg...:] Echo msg (the rest of the line).
  This is useful for displaying progress messages within {\Jaza} scripts.
\item[\% comment:] All lines that start with a percent are ignored.
  This allows you to put comments in your {\Jaza} scripts.
\end{description}
Note that commands like \texttt{eval} and \texttt{evalp} that take an
expression or a predicate as input can be spread over multiple input lines.
If the first line contains more opening brackets than closing, then \Jaza\
continues reading input lines until the number of opening and closing
brackets match. 
\end{document}

%%% Local Variables: 
%%% mode: latex
%%% TeX-master: t
%%% End: 
